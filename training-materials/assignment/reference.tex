% Created 2014-07-20 Sun 17:35
\documentclass[11pt]{article}
\usepackage[utf8]{inputenc}
\usepackage[T1]{fontenc}
\usepackage{fixltx2e}
\usepackage{graphicx}
\usepackage{longtable}
\usepackage{float}
\usepackage{wrapfig}
\usepackage{rotating}
\usepackage[normalem]{ulem}
\usepackage{amsmath}
\usepackage{textcomp}
\usepackage{marvosym}
\usepackage{wasysym}
\usepackage{amssymb}
\usepackage{hyperref}
\tolerance=1000
\setlength{\parindent}{0cm}
\date{\today}
\title{ODE Reference}
\hypersetup{
  pdfkeywords={},
  pdfsubject={},
  pdfcreator={Emacs 24.3.1 (Org mode 8.2.7b)}}
\begin{document}

\maketitle
\tableofcontents

\section{Creating rules}
\label{sec-1}
You can create new rules by clicking the "New" button in the
navigation bar at the top.

\section{Adding features}
\label{sec-2}
You can add features to the LHS of rules by dragging them from the
feature inventory on the left side of the screen and dropping them
on the placeholder that reads "Drop feature here \ldots{}".

To get more information about a feature you can hover over it with
the mouse.

\section{Setting values}
\label{sec-3}
After adding a feature to the LHS of a rule, the system displays a
drop-down menu to the right of the feature that contains a list of
all possible values that this feature can take. By default, feature
values are set to \texttt{underspecified}.

You can change the current value of a feature by expanding the
drop-down menu and selecting the desired value from the list.

\section{Removing features}
\label{sec-4}
You can remove features from the LHS of a rule by clicking the \texttt{x}
button that appears when hovering over their names with the mouse.

\section{Renaming rules}
\label{sec-5}
You can rename a rule by

\begin{enumerate}
\item double-clicking its name
\item entering the new name into the text input field that appears
\item clicking the "OK" button
\end{enumerate}

\section{Changing descriptions}
\label{sec-6}
You can change the description of a rule by

\begin{enumerate}
\item double-clicking it
\item entering the new description into the text input field that
appears
\item clicking the "OK" button
\end{enumerate}

\section{Switching between LHS and RHS}
\label{sec-7}
To switch from the LHS of the current rule to the RHS you can click
the "OutputBuilder" button in the navigation bar. Similarly, to
switch from the RHS back to the LHS you can click the "InputBuilder"
button in the navigation bar.

\section{Adding output strings}
\label{sec-8}
You can add a new output string to a rule by

\begin{enumerate}
\item clicking the placeholder that reads "Add more content"
\item entering the output string
\item clicking the "Add" button
\end{enumerate}

\section{Modifying output strings}
\label{sec-9}
You can modify individual output strings by

\begin{enumerate}
\item double-clicking any word in the string
\item modifying the contents of the text input field that appears
\item clicking the "OK" button
\end{enumerate}

\section{Removing output strings}
\label{sec-10}
You can remove individual output strings from an RHS by clicking the
\texttt{x} button that appears when hovering over them with the mouse.

\section{Splitting output strings}
\label{sec-11}
To split an output string into two parts you can click the area
\emph{between} the last word that belongs to the left part and the first
word that belongs to the right part.

The left part will be added to "Slot 1" and the right part will be
added to "Slot 2". If "Slot 1" and "Slot 2" do not exist, they will
be created first.

\section{Adding parts}
\label{sec-12}
You can add parts to individual slots by

\begin{enumerate}
\item clicking the placeholder at the bottom of the slot
\item entering the part
\item pressing "Enter"
\end{enumerate}

\section{Showing output}
\label{sec-13}
To show all output strings belonging to a given rule you can click
the "Show output" button.

To dismiss the list of output strings, click the \texttt{x} button in the
top right corner of the list.

\section{Modifying parts}
\label{sec-14}
You can modify individual parts by

\begin{enumerate}
\item double-clicking them
\item modifying the contents of the text input field that appears
\item pressing "Enter"
\end{enumerate}

\section{Removing parts}
\label{sec-15}
You can remove individual parts from a given slot by clicking the
\texttt{x} button that appears when hovering over them with the mouse.

\section{Adding slots}
\label{sec-16}
You can create additional slots by clicking the "Add slot" button.

\section{Removing slots}
\label{sec-17}
You can remove individual slots by clicking the \texttt{x} button that
appears when hovering over their headers. The total number of slots
must not drop below 2, so removing slots will only work if there are
at least three slots.

\section{Using the parts inventory}
\label{sec-18}



Every part that you create manually or by splitting an output string
is added to the parts inventory automatically. The parts inventory
is displayed on the left side of the screen in the OutputBuilder.

You can reuse parts to create additional content for the RHS of a
rule by dragging them from the inventory to one of the various
placeholders:

\begin{enumerate}
\item To add a part as an output string you can drop it on the
placeholder that reads "Add more content \ldots{}".

\item To \emph{extend} an existing output string you can drop parts from
the inventory on the placeholder displayed to the right of the
string.

\item To add a part to a specific slot you can drop it on the
placeholder at the bottom of the slot.
\end{enumerate}

\section{Working with multiple groups}
\label{sec-19}
\subsection{Adding groups}
\label{sec-19-1}
You can add a new group to an RHS by clicking the \texttt{+} button in the
header of any existing group.

\subsection{Removing groups}
\label{sec-19-2}
You can remove a group from an RHS by clicking the \texttt{x} button in
the group's header.

\subsection{Copying groups}
\label{sec-19-3}
To use the contents of a given group as a starting point for
creating more content in another group you can click the button
between the \texttt{+} and \texttt{x} buttons in the header of the group you want
to copy.

\section{Browsing rules}
\label{sec-20}
To view a list of all existing rules you can click the "Browse"
button in the navigation bar.

From this interface you can

\begin{itemize}
\item click an entry to view a read-only version of the LHS and RHS of
the corresponding rule
\item use the controls that appear when hovering over a given entry to
\begin{itemize}
\item view a list of similar rules (i.e., rules whose LHS contain
similar features)
\item jump to the InputBuilder
\item jump to the OutputBuilder
\item delete the corresponding rule
\end{itemize}
\item use the "Filter \ldots{}" field to filter entries by name and
description
\end{itemize}
% Emacs 24.3.1 (Org mode 8.2.7b)
\end{document}